%%%%%%%%%%%%%%%%%%%%%%%%%%%%%%%%%%%%%%%%%
% University/School Laboratory Report
% LaTeX Template
% Version 3.1 (25/3/14)
%
% This template has been downloaded from:
% http://www.LaTeXTemplates.com
%
% Original author:
% Linux and Unix Users Group at Virginia Tech Wiki 
% (https://vtluug.org/wiki/Example_LaTeX_chem_lab_report)
%
% License:
% CC BY-NC-SA 3.0 (http://creativecommons.org/licenses/by-nc-sa/3.0/)
%
%%%%%%%%%%%%%%%%%%%%%%%%%%%%%%%%%%%%%%%%%

%----------------------------------------------------------------------------------------
%	PACKAGES AND DOCUMENT CONFIGURATIONS
%----------------------------------------------------------------------------------------

\documentclass{article}

\usepackage[version=3]{mhchem} % Package for chemical equation typesetting
\usepackage{siunitx} % Provides the \SI{}{} and \si{} command for typesetting SI units
\usepackage{graphicx} % Required for the inclusion of images
\usepackage{natbib} % Required to change bibliography style to APA
\usepackage{amsmath} % Required for some math elements 

\setlength\parindent{0pt} % Removes all indentation from paragraphs

\renewcommand{\labelenumi}{\alph{enumi}.} % Make numbering in the enumerate environment by letter rather than number (e.g. section 6)

%\usepackage{times} % Uncomment to use the Times New Roman font

%----------------------------------------------------------------------------------------
%	DOCUMENT INFORMATION
%----------------------------------------------------------------------------------------

\title{Informatics Large Practical} % Title

\author{Lorenzo Baldini} % Author name

\date{\today} % Date for the report

\begin{document}

\maketitle % Insert the title, author and date


% If you wish to include an abstract, uncomment the lines below
% \begin{abstract}
% Abstract text
% \end{abstract}



\pagebreak

%----------------------------------------------------------------------------------------
%	TABLE OF CONTENTS
%----------------------------------------------------------------------------------------

\tableofcontents
\pagebreak

%----------------------------------------------------------------------------------------
%	TABLE OF CONTENTS
%----------------------------------------------------------------------------------------

\section{Software Architecture Description}

This section provides a description of the software architecture of your application. your application is made up of a collection of Java classes; explain why you identified \textit{these classes} as being the right ones for your application. Identify class hierarchical relationships between classes: which classes are subclasses of others?\\[1]


\subsection{UML Diagram}
\label{UML Diagram}
\begin{description}
    \begin{center}
        
    \end{center}
\end{description} 

\subsection{Reasoning}
\label{Reasoning}
\begin{description}
\item[Stoichiometry]
The relationship between the relative quantities of substances taking part in a reaction or forming a compound, typically a ratio of whole integers.
\item[Atomic mass]
The mass of an atom of a chemical element expressed in atomic mass units. It is approximately equivalent to the number of protons and neutrons in the atom (the mass number) or to the average number allowing for the relative abundances of different isotopes. 
\end{description} 


\subsection{Class Relationships}
\label{Class Relationships}
\begin{description}
\item[Stoichiometry]
The relationship between the relative quantities of substances taking part in a reaction or forming a compound, typically a ratio of whole integers.
\item[Atomic mass]
The mass of an atom of a chemical element expressed in atomic mass units. It is approximately equivalent to the number of protons and neutrons in the atom (the mass number) or to the average number allowing for the relative abundances of different isotopes. 
\end{description} 
 
 \pagebreak
%----------------------------------------------------------------------------------------
%	SECTION 2
%----------------------------------------------------------------------------------------

\section{Class Documentation}

\subsection{App.java}
\label{definitions}
\begin{description}
\item[main()]
The main function of this class parses the input arguments and uses the arguments to call the methods playStateful() and playStateless() appropriately.
\item[Atomic mass]
The mass of an atom of a chemical element expressed in atomic mass units. It is approximately equivalent to the number of protons and neutrons in the atom (the mass number) or to the average number allowing for the relative abundances of different isotopes. 
\end{description} 

\subsection{Direction.java}
\label{Direction.java}
\begin{description}
DESCRIBE CLASS
\end{description} 

\subsection{Position.java}
\label{Position.java}
\begin{description}
DESCRIBE CLASS
\end{description} 

\subsection{JsonParser.java}
\label{JsonParser.java}
\begin{description}
DESCRIBE CLASS
\end{description} 

\subsection{Station.java}
\label{Station.java}
\begin{description}
DESCRIBE CLASS
\end{description} 

\subsection{Drone.java}
\label{Drone.java}
\begin{description}
DESCRIBE CLASS
\end{description} 

\subsubsection{StatelessDrone.java}
\label{StatelessDrone.java}
\begin{description}
DESCRIBE CLASS
\end{description} 

\subsubsection{StatefulDrone.java}
\label{StatefulDrone.java}
\begin{description}
DESCRIBE CLASS
\end{description} 
\pagebreak
%----------------------------------------------------------------------------------------
%	SECTION 3
%----------------------------------------------------------------------------------------

\section{Stateful Drone Strategy}

\subsection{Brief description of the stateful drone}
\label{Class Relationships}
\begin{description}
\item[Stoichiometry]
The relationship between the relative quantities of substances taking part in a reaction or forming a compound, typically a ratio of whole integers.
\item[Atomic mass]
The mass of an atom of a chemical element expressed in atomic mass units. It is approximately equivalent to the number of protons and neutrons in the atom (the mass number) or to the average number allowing for the relative abundances of different isotopes. 
\end{description} 

\subsection{Getting positive stations}
\label{Class Relationships}
\begin{description}
\item[Stoichiometry]
The relationship between the relative quantities of substances taking part in a reaction or forming a compound, typically a ratio of whole integers.
\item[Atomic mass]
The mass of an atom of a chemical element expressed in atomic mass units. It is approximately equivalent to the number of protons and neutrons in the atom (the mass number) or to the average number allowing for the relative abundances of different isotopes. 
\end{description} 

\subsection{Moving towards the chosen positive station}
\label{Class Relationships}
\begin{description}
The mass of an atom of a chemical element expressed in atomic mass units. It is approximately equivalent to the number of protons and neutrons in the atom (the mass number) or to the average number allowing for the relative abundances of different isotopes. 
\end{description} 


\subsection{Handling obstacles}
\label{Class Relationships}
\begin{description}
The relationship between the relative quantities of substances taking part in a reaction or forming a compound, typically a ratio of whole integers.
\end{description} 


\subsubsection{Play area boundaries}
\begin{description}
The relationship between the relative quantities of substances taking part in a reaction or forming a compound, typically a ratio of whole integers.
\end{description} 

\subsubsection{Red stations}
\begin{description}
The relationship between the relative quantities of substances taking part in a reaction or forming a compound, typically a ratio of whole integers.
\end{description} 

\subsubsection{Loops}
\begin{description}
The relationship between the relative quantities of substances taking part in a reaction or forming a compound, typically a ratio of whole integers.
\end{description} 


\subsection{Improvement from the stateless drone}
\label{Class Relationships}
\begin{description}
\item[Stoichiometry]
The relationship between the relative quantities of substances taking part in a reaction or forming a compound, typically a ratio of whole integers.
\item[Atomic mass]
The mass of an atom of a chemical element expressed in atomic mass units. It is approximately equivalent to the number of protons and neutrons in the atom (the mass number) or to the average number allowing for the relative abundances of different isotopes. 
\end{description} 


\subsection{Testing}
\label{Class Relationships}
\begin{description}
\item[Stoichiometry]
The relationship between the relative quantities of substances taking part in a reaction or forming a compound, typically a ratio of whole integers.
\item[Atomic mass]
The mass of an atom of a chemical element expressed in atomic mass units. It is approximately equivalent to the number of protons and neutrons in the atom (the mass number) or to the average number allowing for the relative abundances of different isotopes. 
\end{description} 

\pagebreak

%----------------------------------------------------------------------------------------
%	BIBLIOGRAPHY
%----------------------------------------------------------------------------------------

\bibliographystyle{apalike}

\bibliography{bib}

%----------------------------------------------------------------------------------------


\end{document}